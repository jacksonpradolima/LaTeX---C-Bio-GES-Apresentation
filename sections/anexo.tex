
%% colocar um slide com o bla bla bla

\frame{
\begin{block}{}
\SetAlFnt{\tiny}
\begin{algorithm}[H]
\caption{Arquivador baseado em Dominância e Decomposição}
 \SetKwFunction{adicionar}{adicionar}\SetKwFunction{remover}{remover}
  \SetKwProg{madicionar}{}{}{fim}
  \madicionar{\adicionar{nova solução}}{
   adiciona a nova solução\;
 \eSe{todas as soluções forem não dominadas}{
    \remover{}\;
 }{% else
   \eSe{o último nível de dominância só tem uma solução}{
     \eSe{ essa solução pertence a uma região isolada }{
	\remover{}\;
     }{ %else
       remove a solução do último nível\;
     }
   }{%else
     entre as regiões associadas a pelo menos\\
     uma solução 
     do último nível encontre a região mais coberta\;
     \eSe{a região mais coberta tem mais de uma solução}{
	encontra a pior solução da região (por PBI)\;
	remove a pior solução\;
     }{%else
	\remover{}\;
     }
   }
 }
   \remover{}\;
  }{}
  \setcounter{AlgoLine}{0}
  \SetKwProg{mremover}{}{}{fim}
  \mremover{\remover{}}{
    encontra a região mais coberta (desempate por PBI)\;
    encontra a pior solução da região (por PBI)\;
    remove a pior solução\;
  }
\end{algorithm}
\end{block}
}

\frame{
\frametitle{AUC}

\begin{block}{Exemplo de atribuição de recompensa por AUC}
  \begin{figure}
	\centering
	    \includegraphics[width = 
0.8\textwidth]{./figuras/auc.eps}
	  \end{figure}
\end{block}

}

\frame{
\frametitle{Exemplo de Função de Decisão + roleta}

  \begin{block}{Exemplo}
    \begin{columns}[c]
    
  \column{0.55\textwidth}

 \begin{table}
 \tiny
    \begin{tabular}{cccc}
\hline
\textbf{iteracao} & \textbf{desempenho} & \textbf{heurística} & \textbf{tempo 
(s)} \\ \hline
1                 & 0.5                 & 3                 & 3                 
 
\\
2                 & 0.7                 & 1                 & 3                 
 
\\
3                 & 2                   & 2                 & 2                 
 
\\
4                 & 0.2                 & 4                 & 1                 
 
\\
5                 & 1                   & 3                 & 2                 
 
\\
6                 & 4                   & 1                 & 3                 
 
\\
7                 & 5                   & 1                 & 4                 
 
\\
8                 & 2.5                 & 1                 & 3                 
 
\\
9                 & 1                   & 1                 & 4                 
 
\\
10                & 3                   & 3                 & 3                 
 
\\
11                & 3                   & 1                 & 4                 
 
\\
12                & 0.4                 & 4                 & 1                 
 
\\
13                & 0.6                 & 2                 & 1                 
 
\\
14                & 1                   & 4                 & 1                 
 
\\
15                & 0.1                 & 3                 & 2                 
 
\\ \hline
\end{tabular}    
 \end{table}

\column{0.45\textwidth}
     \begin{figure}
	\centering
	    \includegraphics[width = 
0.85\textwidth]{./figuras/cftest.eps}
	  \end{figure} 
   
    \end{columns}

      \end{block}

}

