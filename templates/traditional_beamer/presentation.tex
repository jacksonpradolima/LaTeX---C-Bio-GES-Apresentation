%%%%%%%%%%%%%%%%%%%%%%%%%%%%%%%%%%%%%%%%%
% UFPR Presentation
% LaTeX Template
% Version 1.0 (13/06/2015)
%
% Author: Jackson Antonio do Prado Lima
% Site: http://www.inf.ufpr.br/japlima
%
%%%%%%%%%%%%%%%%%%%%%%%%%%%%%%%%%%%%%%%%%

%----------------------------------------------------------------------------------------
%	PACKAGES AND THEMES
%----------------------------------------------------------------------------------------

\documentclass{beamer} 

% The Beamer class comes with a number of default slide themes
% which change the colors and layouts of slides. Below this is a list
% of all the themes, uncomment each in turn to see what they look like.
% For more theme see http://www.hartwork.org/beamer-theme-matrix/

%\usetheme{default}
%\usetheme{AnnArbor}
%\usetheme{Antibes}
%\usetheme{Bergen}
%\usetheme{Berkeley}
%\usetheme{Berlin}
%\usetheme{Boadilla}
%\usetheme{CambridgeUS}
%\usetheme{Copenhagen}
%\usetheme{Darmstadt}
%\usetheme{Dresden}
\usetheme{Frankfurt}
%\usetheme{Goettingen}
%\usetheme{Hannover}
%\usetheme{Ilmenau}
%\usetheme{JuanLesPins}
%\usetheme{Luebeck}
%\usetheme{Madrid}
%\usetheme{Malmoe}
%\usetheme{Marburg}
%\usetheme{Montpellier}
%\usetheme{PaloAlto}
%\usetheme{Pittsburgh}
%\usetheme{Rochester}
%\usetheme{Singapore}
%\usetheme{Szeged}
%\usetheme{Warsaw}

% As well as themes, the Beamer class has a number of color themes
% for any slide theme. Uncomment each of these in turn to see how it
% changes the colors of your current slide theme.

%\usecolortheme{albatross}
%\usecolortheme{beaver}
%\usecolortheme{beetle}
%\usecolortheme{crane}
%\usecolortheme{dolphin}
%\usecolortheme{dove}
%\usecolortheme{fly}
%\usecolortheme{lily}
%\usecolortheme{orchid}
%\usecolortheme{rose}
%\usecolortheme{seagull}
\usecolortheme{seahorse}
%\usecolortheme{whale}
%\usecolortheme{wolverine}

\definecolor{light_green}{RGB}{230,255,230} % Create a background color
\definecolor{tbGreen}{RGB}{166,210,166}

\setbeamercolor{background canvas}{bg=light_green} % Apply the background color
%\useinnertheme{circles} %define os numeradores com o padrão circles
%\setbeamercolor{block title}{fg=white,bg=green!40!black} %define os titulos dos blocos como verde
%\setbeamercolor{itemize item}{fg=green!40!black} %define os marcadores de ítem como verdes
%\setbeamercolor{itemize subitem}{fg=green!40!black} %define os marcadores de sub-ítem como verdes
%\setbeamercolor{item projected}{fg=white,bg=green!40!black} %define os numeradores como verdes

\setbeamercolor{section in toc shaded}{fg=black} %define os links no tableofcontents como preto (quando estão selecionados)
\setbeamercolor{section in toc}{fg=black} %define os links no tableofcontents como preto (quando não estão selecionados)

\setbeamercolor{structure}{bg=black,fg=green!50!black}
%\setbeamercolor{title}{fg=black, bg=green!40!black}
%\setbeamercolor{frametitle}{fg=black, bg=green!80!black}


\setbeamertemplate{navigation symbols}{} % remover barra de navega‹o
%\setbeamertemplate{footline}[frame number] % numero de paginas

\setbeamertemplate{footline}
{
	\leavevmode%
	\hbox{%
		\begin{beamercolorbox}[wd=.333333\paperwidth,ht=2.25ex,dp=1ex,center]{author in head/foot}%
			\usebeamerfont{author in head/foot}\insertshortinstitute
		\end{beamercolorbox}%
	    \begin{beamercolorbox}[wd=.333333\paperwidth,ht=2.25ex,dp=1ex,center]{title in head/foot}%
			\usebeamerfont{title in head/foot}
		\end{beamercolorbox}%
		\begin{beamercolorbox}[wd=.333333\paperwidth,ht=2.25ex,dp=1ex,right]{date in head/foot}%
			\usebeamerfont{date in head/foot}\insertshortdate{}\hspace*{2em}
			\insertframenumber{} / \inserttotalframenumber\hspace*{2ex} 
		\end{beamercolorbox}
	}%
	\vskip0pt%
}

\usepackage[brazil]{babel} %Allow the use of brazilian portuguese language
%\usepackage[english]{babel}
%\usepackage[utf8x]{inputenc} % acento normal
\usepackage[utf8]{inputenc}

\usepackage{scalefnt}
\usepackage{color, colortbl, multirow}
\usepackage{epstopdf} % usar figura eps
\usepackage{verbatim} % para colocar o comment
\usepackage[portuguese,ruled,lined]{algorithm2e}
\usepackage{algorithmic}

\algsetup{linenosize=\tiny}

\setcounter{secnumdepth}{5}
\SetKwFor{Para}{para}{fa\c{c}a}{fim para}
\SetKwFor{ParaCada}{para cada}{fa\c{c}a}{fim para cada}
\SetKwBlock{Inicio}{inicio}{fim}

\usepackage{graphicx} % Allows including images
\usepackage{booktabs} % Allows the use of \toprule, \midrule and \bottomrule in tables

\usepackage[export]{adjustbox}

\hypersetup{pdfpagemode=FullScreen}   % Full Screen
\hypersetup{pdfpagelayout=SinglePage} % Page layout

\pgfdeclareimage[height=0.5cm]{c-bio}{figuras/logo.png}
\pgfdeclareimage[height=0.5cm]{ufpr}{figuras/ufpr.png}
\logo{\pgfuseimage{ufpr} \hspace{320pt} \pgfuseimage{c-bio}}


%%===================== Capa =========================%%


\title[Short title]{\textbf{Título}} % The short title appears at the bottom of every slide, the full title is only on the title page

\author{Autor 1 \\ Autor 2} 
\institute[UFPR - Universidade Federal do Paraná] % Your institution as it will appear on the bottom of every slide, may be shorthand to save space
{
	Universidade Federal do Paraná \\ % Your institution for the title page
	Programa de Pós-Graduação em Informática
}

\date{\today} % Date, can be changed to a custom date. You can use \day \month \year or \today, for example.

\begin{document} % inicio do documento

\frame{
	\titlepage
}

%%===========================ROTEIRO=============================%%
\frame{
\frametitle{Agenda}

   \hspace*{1cm}
	\begin{minipage}[t][6cm][t]{\textwidth}
		\vspace{-30pt}
		\tableofcontents
	\end{minipage}
}
%%========================Introducao================================%%
\section{Introdução}

%%========================Objetivo================================%%
\subsection{Objetivo}
\frame{
	\frametitle{Objetivo}
	
	\begin{block}{Objetivo Geral}
		Texto
	\end{block}
}
%%===================Método proposto==============%%
\section{Método Proposto}
\subsection{Método Proposto}
\frame{
  \frametitle{Método Proposto}
   
   \begin{block}{Texto}
      	\begin{itemize}
      	 \item Texto
      	 \item Texto
      	\end{itemize}
   \end{block}
}

\subsection{Objetivos}
\frame{
	\frametitle{Objetivos}
	
	\begin{block}{Objetivos do trabalho}
			\begin{itemize}
				\item Texto;
				\item Texto;
				\item Texto;
				\item Texto;
				\item Texto.
			\end{itemize}
	\end{block}
	
	\begin{block}{Funções objetivas do trabalho}
		\begin{itemize}
			\item $ maxf_{i}(T)=pc(T) $
			\item $ minf_{ii}(T)=\frac{custo(T)}{nº \; c \; de \; t} $
			\item $ maxf_{iii}(T)=\frac{pref(T)}{nº \; c \; de \; t} $
			\item $ maxf_{iv}(T)=escore(T) $
			\item $ minf_{v}(T)=nCasos(T) $
		\end{itemize}
	\end{block}
}

\frame{
	\frametitle{Funções objetivas do trabalho}
	
	\begin{block}{Funções objetivas do trabalho}
		\begin{itemize}
			\item $ pc(T)=\frac{(nº \; de \; p \; c)}{(nº \; de \; p \; c)}  $
			\item $ c(T)=\displaystyle\sum_{i=0}^{i<n} c(p_{i}) $
			\item $ p(T)=\displaystyle\sum_{i=0}^{i<n} p(p_{i}) $	
			\item $ score(T)=\frac{(nº \; de \; m \; m)}{(nº \; de \; m \; g + nº \; de \; m \; e)} $
			\item $ nCasos(T)=\frac{(nº \; de \; c \; de \; t)}{(total \; de \; p)}  $
		\end{itemize}
	\end{block}
}

\frame{
	\frametitle{\textit{Tabela}}

	\begin{block}{\textit{Tabelas} utilizadas}
	\begin{table}[!htb]
		\renewcommand{\arraystretch}{1.5}
		\fontsize{10pt}{12pt}\selectfont
		\centering
		\scalebox{0.8}{
			\begin{tabular}{c | c | c | c | c }
				\toprule
				\textbf{Matriz} & \textbf{Qtde de P} & \textbf{Qtde de M} & \textbf{Qtde de Pa} & \textbf{Qtde de C}\\ \midrule
				texto1 & 450 & 227 & 183 & 21\\ \midrule
				texto2 & 1152 & 394 & 202 & 22\\ \midrule
				texto3 & 68 & 106 & 75 & 14\\ \midrule
				texto4 & 504 & 357 & 195 & 22\\
				\bottomrule
			\end{tabular}
		}
	\end{table}
	\end{block}
}	
%%===================Resultados==============%%
\section{Configuração dos Experimentos}
\subsection{Experimentos}
\frame{
	\frametitle{Experimentos}
	\begin{block}{Representação da População de Exemplo}
	\begin{table}[!htb]
		\renewcommand{\arraystretch}{1.5}
		\fontsize{10pt}{12pt}\selectfont
		\label{table:population}
		\centering
		\begin{tabular}{c | c | c | c | c | c }
			\toprule
			\textbf{$P_{1}$} & \textbf{$P_{2}$} & \textbf{$P_{3}$} & \textbf{$P_{4}$} & ... & \textbf{$P_{n}$}\\ \midrule
			0 & 1 & 1 & 0 & ... & ...\\
			\bottomrule
		\end{tabular}
		\end{table}
	\end{block}
}

\frame{
	\frametitle{Configuração dos Experimentos}
	\begin{block}{Texto}
		Texto.
	\end{block}
	
	\begin{block}{Equação}
		\begin{equation}
			H = \left(\begin{array}{c} M + p -1 \\ p \end{array}\right)
		\end{equation}
	\end{block}
}

\frame{
	\frametitle{Configuração dos Experimentos}

	\begin{block}{Texto}
		\begin{table}[!h]
			\renewcommand{\arraystretch}{1.5}
			\fontsize{10pt}{12pt}\selectfont
			\centering
			\begin{tabular}{ c | c | c | c }
				\toprule
				\textbf{D} & 4 & 5 & 6 \\ \midrule
				\textbf{R} & 70 & 126 & 210 \\ \midrule
				\textbf{P} & 70 & 126 & 210 \\ 
				\bottomrule
			\end{tabular}
		\end{table}
	\end{block}
}

\frame{
	\frametitle{Configuração dos Experimentos}
	
	\begin{block}{Operadores, Probabilidades e Gerações}
		\begin{table}[!h]
			\renewcommand{\arraystretch}{1.5}
			\fontsize{10pt}{12pt}\selectfont
			\centering
			\begin{tabular}{ l | c }
				\toprule
				\textbf{Parâmetro} & \textbf{Valores} \\ \midrule
				Cruzamento & \textit{Single Point} e \textit{Uniform} \\ \midrule
				Mutação & \textit{Bit Flip} e \textit{Product Swap} \\ \midrule
				Seleção & Torneio Binário \\ \midrule
				Probabilidade de Cruzamento &  80\% e 90\% \\ \midrule
				Probabilidade de Mutação & 0.5\% e 1\%\\ \midrule
				Quantidade de Gerações & 200, 400 e 600 \\
				\bottomrule
			\end{tabular}
		\end{table}
	\end{block}
}

\frame{
	\frametitle{Configuração dos Experimentos}
	
	\begin{block}{Quantidade de Avaliações}
		\begin{table}[!h]
			\renewcommand{\arraystretch}{1.5}
			\fontsize{10pt}{12pt}\selectfont
			\centering
			\begin{tabular}{c | c | c | c | c }
				\toprule
				\multirow{5}{*}{\textbf{Geração}} & \multicolumn{4}{c}{\textbf{População}} \\
				\cline{2-5}
				& & \textbf{70} & \textbf{126} & \textbf{210} \\
				\cline{2-5}
				& \textbf{200} & 14.000 & 25.200 & 42.000 \\
				\cline{2-5}
				& \textbf{400} & 28.000 & 50.400 & 84.000 \\
				\cline{2-5}
				& \textbf{600} & 42.000 & 75.600 & 126.000 \\
				\bottomrule
			\end{tabular}
		\end{table}
	\end{block}
}

\section{Resultados}		
\subsection{Análise}
\frame{
	\frametitle{Análise dos Resultados}
	
	\begin{block}{Métodos}
		\begin{itemize}
			\item Análise da Fronteira de Pareto\;
			\item Análise do Hypervolume\;
			\item Teste estatístico de Kruskal Wallis.
		\end{itemize}
	\end{block}
}

\subsection{Resultados}
\frame{
	\frametitle{Melhor Configuração}	
	\begin{table}[!htb]
		\fontsize{10pt}{12pt}\selectfont
		\centering
		\scalebox{0.8}{
			\begin{tabular}{c | c | c | c | c | c | c | c | c}
				\toprule
				\textbf{Matriz} & \textbf{Algoritmo} & \textbf{Aval.} & \textbf{Pop.} & \textbf{Ger.} & \textbf{\%Cruz} & \textbf{\%Mut} & \textbf{Cruz.} & \textbf{Mut.}\\
				\midrule
				\multirow{2}{*}{c} & NSGA-II & 25.200 & 126 & 200 & 90\% & 1\%  & Unif. & PS \\
				\cline{2-9} & NSGA-III & 14.000 & 70 & 200 & 90\% & 0.5\% & Unif. & PS \\
				\midrule
				\multirow{2}{*}{e} & NSGA-II & 25.200 & 126 & 200 & 90\% & 1\% & Unif. & PS \\
				\cline{2-9} & NSGA-III & 25.200 & 126 & 200 & 90\% & 1\% & Unif. & PS \\
				\midrule
				\multirow{2}{*}{j} & NSGA-II & 25.200 & 126 & 200 & 80\% & 1\% & SP & PS \\
				\cline{2-9} & NSGA-III & 14.000 & 70 & 200 & 90\% & 0.5\% & SP &  BF \\
				\midrule
				\multirow{2}{*}{w} & NSGA-II & 14.000 & 70 & 200 & 90\% & 1\%  & Unif. & PS \\
				\cline{2-9} & NSGA-III & 14.000 & 70 & 200 & 90\% & 0.5\% & Unif. & BF \\
				\bottomrule
			\end{tabular}
		}
	\end{table}
}

\frame{
	\frametitle{Resultados encontrados pelos experimentos}	
	\begin{table}[!hbt]
		\renewcommand{\arraystretch}{1.5}
		\fontsize{10pt}{12pt}\selectfont
		\centering
		\scalebox{0.75}{
			\begin{tabular}{c | c | c | c | c | c | c }
				\toprule
				\textbf{Matriz} & \textbf{Algoritmo} & \textbf{Média (HV)} & \textbf{DP} &  \textbf{Maior (HV)} & \textbf{DP} & \textbf{Kruskal} \\
				\midrule
				\multirow{2}{*}{c} & NSGA-II & \textbf{0.62023} & 0.04014 & 0.46567 & 0.00442 & \multirow{2}{*}{TRUE} \\
				\cline{2-6} & NSGA-III & 0.54753 & 0.03500 & 0.43851 & 0.01429 & \\
				\midrule
				\multirow{2}{*}{e} & NSGA-II & \textbf{0.62477} & 0.03647 & 0.63374 & 0.01673 & \multirow{2}{*}{FALSE} \\
				\cline{2-6} & NSGA-III & \textbf{0.64214} & 0.03435 & 0.64977 & 0.02570 & \\
				\midrule
				\multirow{2}{*}{j} & NSGA-II & \textbf{0.46720} & 0.011003 & 0.51034 & 0.00812 & \multirow{2}{*}{TRUE} \\
				\cline{2-6} & NSGA-III & 0.41669 & 0.02091 & 0.47553 & 0.01359 & \\
				\midrule
				\multirow{2}{*}{w} & NSGA-II & 0.55486 & 0.03767 & 0.50344 & 0.00970 & \multirow{2}{*}{TRUE} \\
				\cline{2-6} & NSGA-III & \textbf{0.57406} & 0.031040 & 0.46240 & 0.014059 & \\
				\bottomrule
			\end{tabular}
		}
	\end{table}		
}

\frame{
	\frametitle{Quantidade de soluções nas Fronteiras de Pareto}
	
	\begin{table}[!htb]
		\renewcommand{\arraystretch}{1.5}
		\fontsize{10pt}{12pt}\selectfont
		\centering
		\begin{tabular}{c | c | c | c }
			\toprule
			\multirow{2}{*}{\textbf{Matriz}} & \multirow{2}{*}{$ PF_{aprox}$ } & \multicolumn{2}{c}{\textbf{$ PF_{know} (PF_{true} / \%) $}} \\
			\cline{3-4} &  & NSGA-II & NSGA-III \\
			\midrule
			c & 214 & \textbf{189(181 / 84,58\%)} & 116(33 / 15,42\%) \\
			\midrule
			e & 119 & 118(59 / 49,58\%) & \textbf{96(60 / 50,42\%)} \\
			\midrule
			j & 590 & \textbf{542(361 / 61,19\%)} & 331(229 / 38,81\%) \\
			\midrule
			w & 113 & 95(53 / 46,90\%) & \textbf{89(60 / 53,1\%)} \\
			\bottomrule
		\end{tabular}
	\end{table}
}

\frame{
	\frametitle{Tempo de execução}
	
	\begin{table}[!htb]
		\renewcommand{\arraystretch}{1.5}
		\fontsize{10pt}{12pt}\selectfont
		\centering
		\begin{tabular}{c | c | c | c }
			\toprule
			\textbf{Matriz} & \textbf{Algoritmo} & \textbf{Tempo Total (s)} & \textbf{Média (s)}\\
			\midrule
			\multirow{2}{*}{c} & NSGA-II & 250 & 8.3 \\
			\cline{2-4} & NSGA-III & \textbf{161} & \textbf{5.4} \\
			\midrule
			\multirow{2}{*}{e} & NSGA-II & \textbf{1.457} &  \textbf{48} \\
			\cline{2-4} & NSGA-III & 1.623 & 54 \\
			\midrule
			\multirow{2}{*}{j} & NSGA-II & 46 & 1.5 \\
			\cline{2-4} & NSGA-III &  \textbf{32} & \textbf{1} \\
			\midrule
			\multirow{2}{*}{w} & NSGA-II & \textbf{220}  & \textbf{7} \\
			\cline{2-4} & NSGA-III & 247 & 8 \\
			\bottomrule
		\end{tabular}
	\end{table}
}

\section{}
\begin{frame}
	\Huge{\centerline{Obrigado!}}
\end{frame}

\end{document} 
